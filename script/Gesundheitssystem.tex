\section{Grundlagen des Gesundheitssystems}

Das Gesundheitssystem ist ``die Gesamtheit eines organisierten Handelns als Antwort auf das Auftreten von Krankheit und Behinderung und zur Abwehr gesundheitlicher Gefahren''\footnote{Schwartz u. Busse 2003, S. 519}. Da es sehr viele Möglichkeiten gibt, diese Ziele zu erreichen, und den vielen relevanten Faktoren dabei unterschiedliche Wichtigkeiten zugemessen werden können, sind Gesundheitssysteme auf der Welt sehr unterschiedlich ausgeprägt. Beim Gesundheitssystem in Deutschland steht dabei eine Sicherstellung der adäquaten medizinischen Versorgung aller Bürger, die Minimierung der sich aus Krankheit, Behinderung oder Unfall ergebenden Nachteile für den Einzelnen, sowie die individuelle Freiheit (in diesem Kontext insbesondere ausgeprägt in der Freiheit der Wahl der Diestleister für medizinische Versorgung sowie der Krankenkasse) im Fokus. Entsprechend ist eine zentrale Aufgabe des Gesundheitssystems in Deutschland die Sicherstellung einer möglichst gerechten Finanzierung der medizinischen Ausgaben - was nur bei Einhaltung einer kosteneffizienten Arbeitsweise möglich ist.

Zu diesem Zweck arbeiten medizinisches Personal, Krankenkassen, Interessenvertretungen und Behörden in einem komplexen System zusammen. Auf Seiten der Regierung ist das \textbf{Bundesministerium für Gesundheit} (\textbf{BMG}) dabei der wichtigste Akteur. Dieses wird unter Anderem von dem \textbf{Robert Koch-Institut} (\textbf{RKI}) und von dem \textbf{Bundesinstitut für Arzneimittel und Medizinprodukte} (\textbf{BfArM}) beraten. Aufgabe des RKI ist dabei in erster Linie die Forschung an Krankheitserregern, die Unterstützung bei der Aufklärung von Krankheitsausbrüchen und die Untersuchung von Epidemien und Zivilisationskrankheiten. Basierend auf diesen Arbeiten erarbeitet das RKI unter Anderem Empfehlungen zu Notfallplänen, Impfvorsorge\footnote{Ständige Impfkommission}, Krankenhaushygienevorschriften etc., aus denen sich Änderungen an der Gesetzgebung ergeben. Das BfArM befasst sich vor Allem mit der Zulassung und Registrierung von Arzneimitteln sowie der Risikoabschätzung und -Bewertung von Arzneimitteln und Medizinprodukten. Dabei lässt es Ergebnisse aus der eigenen Forschung in diesen Bereichen einfließen. Basierend auf den Ergebnissen seiner Arbeiten unterstützt es ebenfalls das BMG beispielsweise bei der Gesetzgebung oder bei der Abwehr akuter Gefahren für die Gesundheit der Bevölkerung. 

Gemeinsam setzen die Behörden so durch Gesetzgebung und Verordnungen den allgemeinen Rahmen für die medizinische Versorgung. Dieser Rahmen muss sich allerdings außer an den ermittelten Berüfnissen und aktuellen Forschungsergebnissen auch an der Realität des medizinischen Personals, welches die Versorgung durchführt, orientieren. Entsprechend werden viele Entscheidungen über die konkrete Ausgestaltung, wie z.B. über die Zulassung von neuen Untersuchungs- oder Behandlungsmethoden, vom \textbf{gemeinsamen Bundesausschuss} getroffen, dem hochrangige Vertreter der Krankenkassen, Ärzte, Zahnärzte und Psychotherapeuten angehören. Unter Anderem auf Basis dieser Entscheidungen handeln die einzelnen Interessenverbände - die \textbf{Kassenärztliche Bundesvereinigung}, der \textbf{Deutsche Apothekerverband} sowie die \textbf{Deutsche Krankenhausgesellschaft} - Verträge mit dem Interessenverbund der Krankenkassen - dem \textbf{Spitzenverband Bund der Krankenkassen} - bezüglich der Vergütung für die einzelnen Elemente der medizinischen Versorgung aus. Dabei wird ein Mittelweg zwischen einer optimalen medizinischen Versorgung und den verfügbaren finanziellen Mitteln, die bei den Krankenkassen primär aus den Beiträgen der ca. 73 Millionen Pflichtversicherten sowie aus Zuschüssen aus dem Bundeshaushalt stammen, gefunden. 

Dieses System wird zunehmend durch den demografischen Wandel und die technologische Entwicklung belastet. Neue Forschungsergebnisse führen zu der Verfügbarkeit immer besserer, aber z.T. auch immer teurerer, Behandlungsoptionen. Zudem steigt die Anzahl an vergleichsweise trivialen Erkrankungen, die früher tödlich verlaufen wären, die aber nun behandelbar sind. Resultat ist eine Reduktion von Todesfällen aufgrund einer kurzen Krankheit und ein Anstieg chronischer oder zumindest über lange Zeit hinweg beherrschbarer Krankheiten, was zu einem starken Anstieg der Kosten im Gesundheitswesen führt. Ein weiterer Effekt ist ein zunehmend ungedeckter, steigender Bedarf an medizinischem Personal, da die Anzahl an Patienten mit komplexen Erkrankungen, deren Behandlung mehr und intensivere Patientenkontakte erfordert, steigt. All dies sind Beispiele für den Erfolg des Gesundheitssystems: Die genannten Probleme ergeben sich aus der erfolgreichen Antwort auf das Auftreten von Krankheit und Behinderung und die Abwehr gesundheitlicher Gefahren. Sie fordern aber eine adäquate Weiterentwicklung des Systems, um weiterhin eine erfolgreiche Bewältigung der gestellten Aufgaben zu ermöglichen.

Die Gesundheitsinformatik kann dabei an vielen Stellen unterstützend wirken. Beispielsweise kann durch die Entwicklung von mobilen Überwachungssystemen für Vitaldaten, kombiniert mit zuverlässiger automatisierter Auswertung und Erkennung von Gefahren für die Gesundheit sowie daraus resultierender Alarmierung von medizinischem Personal, die Anzahl an trivialen aber zeitaufwändigen Patientenkontakten für Routineuntersuchungen reduziert werden. Dadurch wird sowohl das medizinische Peronal entlastet als auch die Lebensqualität der Patienten durch die Reduktion von Kontrollterminen und die Möglichkeit, schneller aus der stationären Pflege entlassen zu werden, gesteigert. Ein weiteres Beispiel ist eine Verbesserung der Technikintegration. Viele Systeme in Krankenhäusern und Arztpraxen verwenden z.T. veraltete, proprietäre Protokolle und Kommunikationsmechanismen, die effiziente Dokumentation und Datenaustausch unmöglich machen. Dadurch wird die Qualität der Behandlung, die häufig von der Qualität der zur Verfügung stehenden Information abhängt, reduziert, und das medizinische Personal durch ineffiziente, medienbruchreiche, fehleranfällige und monotone Dokumentationstechniken unnötig belastet.

Ein weiteres Gebiet, auf dem die Informatik im Gesundheitswesen unterstützend wirkt und dem sich dises Modul insbesondere widmet, ist die Datenauswertung in der molekularen Diagnostik. Durch den Einsatz moderner Methoden ist es heutzutage möglich, die kompletten Genome von Krankheitserregern zu entschlüsseln. Neben vielen anderen Entwicklungen ermöglicht dies, Krankheitsausbrüche mit einer nie dagewesenen Geschwindigkeit und Genauigkeit zu rekonstruieren und daraus Handlungsoptionen abzuleiten (beispielsweise die zeitweilige Schließung und Reinigung einer Lebensmittelproduktionsstelle oder die Entfernung eines bisher unbekannten Keimherdes in einer Krankenhausstation), die den Ausbruch effizient eindämmen und die Ansteckung weiterer Personen vermeiden. Dabei wird allerdings mit großen Datenmengen gearbeitet, die für menschliche Experten in ihrer Rohform nicht verarbeitbar sind. Entsprechend muss auf Basis eines Verständnisses der biologischen Zusammenhänge und des medizinischen Bedarfs Software entwickelt werden, die die großen Datenmengen auf Darstellungen und Aussagen reduziert, die Experten in der Entscheidungsfindung unterstützen. 


\subsection{Empfohlene Youtube-Videos}

\begin{description}[align=left]
	\item[Überblick Gesundheitssystem] \href{https://www.youtube.com/watch?v=sasX7aZeCGc}{https://www.youtube.com/watch?v=sasX7aZeCGc}
	\item[Demografischer Wandel und Gesundheitssystem] \href{https://www.youtube.com/watch?v=Uqo5vkkT2Qg}{https://www.youtube.com/watch?v=Uqo5vkkT2Qg}
\end{description}

\subsection{Kontrollfragen}
\begin{enumerate}
	\item Was sind die Aufgaben des Gesundheitssystems?
	\item Was ist eine Aufgabe der Behörden im Gesundheitssystem?
	\item Warum sind Interessenverbände des medizinischen Personals wichtig?
	\item Was kann die IT tun, um das Gesundheitssystem zu unterstützen?
\end{enumerate}
