\section{MLST und MST}

Die Sequenzierung kann verwendet werden, um Unterschiede zwischen den Genomen von Krankheitserregern unterschiedlicher Patienten zu identifizieren und anhand der Anzahl an Mutationen, die die Krankheitserreger voneinander unterscheiden, auf Anteckungsreihenfolgen zu schließen und Krankheitsausbrüche zu rekonstruieren\footnote{Beispielsweise, um den Urpsrung des Ausbruchs zu identifizieren und zu beseitigen - die Wichtigkeit slcher Arbeiten wurde unter Anderem beim EHEC-Ausbruch 2011 deutlich.}. Ohne eine vollständige Rekonstruktion des Genoms des Krankheitserregers, die bis vor ca. 20 Jahren prohibitiv teuer war\footnote{Die Rekonstruktion des ersten menschlichen Genoms mittels Sanger-Sequenzierung kostete 1999-2000 ca. 300 Millionen US-Dollar. Ein bakterielles Genom ist ca. 100-fach kleiner als das Menschliche, eine komplette Rekonstruktion der Genome mehrerer potentiell an einem Ausbruch beteiligter Bakterien würde aber entsprechend immer noch Millionen Euro kosten.}, und heute mit modernen Sequenziermethoden immer noch insbesondere bioinformatisch aufwändig ist, lässt sich aber keine exakte Bestimmung aller Mutationen durchführen. Entsprechend wurde eine andere, auch mittels Sanger einfach durchführbare Methode entwickelt: Das \textbf{Multi-Locus Sequence Typing} (\textbf{MLST}).

\subsection{MLST}

Beim MLST werden anstatt vollständiger Genome nur mehrere repräsentative Bereiche (Loci) der Genome zum Vergleich und zur Sequenz-basierten Typisierung von Bakterien verwendet. Dabei werden für eine Bakterien-Spezies mehrere Gene (meist 6-8) identifiziert, die von konservierten Regionen flankiert sind. Diese Gene können aufgrund der flankierenden konservierten Regionen zuverlässig in allen Bakterien dieser Spezies mittels PCR amplifiziert und danach sequenziert werden. Entsprechend kann ein einfacher Vergleich eines definierten kleinen\footnote{Die bei der MLST verwendeten Gene sind meist maximal 2000 Basen lang, entsprechend werden ca. 15 Tausend Basen verglichen. Ein komplettes bakterielles Genom hat eine Länge zwischen ca. 1-10 Millionen Basen.} Teilbereichs der Genome von Bakterien dieser Spezies durchgeführt werden.  

Um eine einfache Nomenklatur und ein einfaches Distanzmaß zu gewinnen wird dabei der Sequenztyp eines Bakteriums nicht auf Basis der exakten Positionen und Arten der vorhandenen Mutationen definiert. Stattdessen werden für die einzelnen Gene \textbf{Alleltypen} definiert, aus denen sich der \textbf{Sequenztyp} zusammensetzt. Der Alleltyp eines Gens wird auf Basis einer Datenbank bestimmt. In dieser Datenbank sind alle unterschiedlichen Sequenzen des Gens, die bisher bekannt sind, nach dem Datum des Hinzufügens zur Datenbank aufsteigend sortiert und durchnummeriert gespeichert. Die Nummer einer Sequenz in der Datenbank ist der Alleltyp. Soll der Alleltyp eines Gens eines Bakteriums der gleichen Spezies aus einer neuen Probe bestimmt werden, wird die Sequenz mit der Datenbank abgeglichen. Ist die Sequenz bereits in der Datenbank vorhanden, entspricht der Alleltyp für dieses Gen der Nummer (also dem Alleltyp) der Sequenz in der Datenbank. Ist für das Gen diese exakte Sequenz noch nicht in der Datenbank vorhanden - dies weist auf eine oder mehrere unbekannte Mutationen hin - wird die Sequenz der Datenbank hinzugefügt, ihr wird die nächste freie Nummer zugewiesen, und der Alleltyp des Gens ist diese Nummer.

Wurde für ein Bakterium für alle Gene der jeweilige Alleltyp bestimmt, wird in einer weiteren Datenbank überprüft, ob diese Kombination von Alleltypen bereits bekannt ist. In dieser Datenbank stehen alle bereits bekannten Kombinationen von Alleltypen, wieder aufsteigend nach dem Zeitpunkt ihres Hinzufügens zur Datenbank sortiert und durchnummeriert. Die Nummer einer Alleltyp-Kombination ist der Sequenztyp. Ist die Kombination der Alleltypen bereits in der Datenbank vorhanden, ist die Nummer der Kombination in der Datenbank (also der Sequenztyp) der Sequenztyp des Bakteriums. Ist die Kombination unbekannt, wird wie beim Alleltyp die neue Alleltyp-Kombination der Datenbank hinzugefügt, bekommt die nächste Nummer und diese Nummer ist der Sequenztyp des Bakteriums.

Die Liste an Genen, die für die Typisierung einer Bakterien-Spezies verwendet wird, wird als \textbf{MLST-Schema} bezeichnet. Diese MLST-Schemen mit den dazugehörigen Datenbanken der bekannten Alleltypen aller Gene des Schemas sowie der bekannten Sequenztypen werden zentral bei \href{https://pubmlst.org/}{https://pubmlst.org} verwaltet. Dadurch wird eine gemeinsame Nomenklatur für den Vergleich unterschiedlicher Bakterien geschaffen - haben zwei Labore beispielsweise einen Vibrio cholerae\footnote{Der bakterielle Erreger der Cholera.} als Sequenztyp 196 klassifiziert, kann davon ausgegangen werden, dass es sich (mit der Auflösung, die MLST ohne komplette Betrachtung des gesamten Genoms erlaubt) um den gleichen Krankheitserreger handelt. 

Neben einer gemeinsamen Nomenklatur schafft die MLST ein zentrales Distanzmaß: Als eine erste Näherung der Unterschiedlichkeit von zwischen zwei Krankheitserregern kann die Anzahl der Gene, in denen der Alleltyp unterschiedlich ist, genommen werden. Bei der klassischen MLST ist die Auflösung dabei allerdings extrem eingeschränkt: Die maximale zwischen zwei Bakterien detektierbare Distanz entspricht der Anzahl der Gene in dem MLST-Schema, also meist 6-8. Um eine längere Infektkette zu rekonstruieren ist diese Zahl unzureichend, und für die Aufklärung zeitlich beschränkter Ausbrüche reicht die Auflösung nicht: Mit hoher Wahrscheinlichkeit unterscheiden sich auch bei einem Ausbruch mit mehreren zig Beteiligten alle Proben nur um 1-2 Alleltypen voneinander, womit eine Differenzierung der Ähnlichkeiten und eine entsprechende Aufklärung von Ansteckungsreihenfolgen nicht möglich ist.  

\subsection{cgMLST}

Seit Mitte der 1990er Jahre entwickelte und seit Mitte der 2000er Jahre zunehmend in der klinischen Forschung und später auch der Routinediagnostik eingesetzte neue Sequenziertechnologien (Next Generation Sequencing, NGS) erlauben es, der Sanger- oder der Pyrosequenzierung sehr ähnliche Reaktionen massiv parallel in kurzer Zeit durchzuführen, so dass sogar mehrere Milliarden Sequenzierungen gleichzeitig durchgeführt werden können. Dadurch kann vergleichsweise preiswert Sequenzinformation aus dem ganzen Genom eines Krankheitserregers generiert werden. Die Rekonstruktion des gesamten Genoms ist dabei allerdings immer noch eine bioinformatisch komplexe Aufgabe, wodurch ein kompletter Vergleich der Genome der an einem Ausbruch beteiligten Bakterien noch immer nicht praktikabel ist. Die Extraktion der Sequenzinformation zu einzelnen Genen aus einem solchen Datensatz ist aber recht trivial. Entsprechend wurde eine Erweiterung des MLST basierend auf NGS-Daten entwickelt: Das \textbf{core-genome MLST} (\textbf{cgMLST}). Dabei wird durch Untersuchung vieler Datensätze von Sequenzierungen von Bakterien einer Spezies das core genome des Bakteriums ermittelt: Eine Liste der Gene, die für Bakterien dieser Spezies essentiell ist und entsprechend in jedem Bakterium dieser Spezies vorhanden sein sollte. Auf Basis dieses core genomes werden Typisierungen auf die gleiche Art, wie bei der klassichen MLST durchgeführt. Die Auflösung ist durch die Verwendung von mehreren Tausend anstatt von unter zehn Genen deutlich höher und zeigt zum Teil bereits bei direkter gegenseitiger Ansteckung zweier Personen unterschiedliche Sequenztypen für die Bakterien in diesen beiden Personen.

\subsection{MST und Krankheitserreger-Ausbrüche}

Um aus cgMLST-Profilen ein Ausbruchsgeschehen zu rekonstruieren, muss die plausibelste Erklärung für die zwischen den untersuchten Krankheitserregern gefundenen Unterschiede gefunden und visualisiert werden. Dies kann mit Hilfe eines \textbf{minimum spanning tree} (\textbf{MST}) erfolgen. Ein MST ist eine solche Auswahl von Kanten eines Graphen, dass alle Knoten miteinander ohne Zyklen und mit dem niedrigsten möglichen summierten Kantengewicht verbunden sind. Der MST eines cgMLST-Graphen, bei dem jeder Knoten ein Bakterium darstellt und das Gewicht einer Kante zwischen zwei Knoten die Distanz zwischen den Sequenztypen der beiden Bakterien ist, zeigt somit einen plausiblen Ansteckungsverlauf: Kurze Kanten deuten potentiell auf direkte Ansteckungen hin, längere Kanten auf Ansteckungen die indirekt - also über weitere Beteiligte oder längere Zeiträume hinweg, für die in dem Datensatz keine Sequenzinformation vorhanden ist - erfolgten. Dies stützt sich auf der Tatsache, dass Mutationen rein zufällig stattfinden, über einen gewissen Zeitraum (wie die typische Dauer zwischen Ansteckung eines Patienten und Ausbruch der Krankheit, die die Weitergabe der Krankheitserreger an weitere Patienten ermöglicht) sich also immer eine ähnliche Anzahl von Mutationen in einem Genom ansammeln sollte. 

Ein MST kann beispielsweise mittels Kruskal's Algorithmus erstellt werden, der aus dem Modul Algorithem und Datenstrukturen bereits bekannt sein sollte. 
